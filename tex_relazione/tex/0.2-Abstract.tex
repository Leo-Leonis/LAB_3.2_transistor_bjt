Lo scopo dell’esperienza è stato quello di misurare la caratteristica I-V di un transistor BJT PNP a emettitore comune. In particolare, abbiamo acquisito le misure della tensione tra collettore ed emettitore e della corrente di collettore per due diversi valori della corrente di base, ossia $I_{B_1} = -50$ \si{\micro A} e $I_{B_2} = -100$ \si{\micro A}.\\
Mediante un fit sono stati poi calcolati la tensione di Early $V_a$ e la resistenza di uscita $R_L$ per ogni curva, da cui è stato possibile ricavare il guadagno $\beta$ e la conduttanza $g$. Per quanto riguarda i valori dei parametri trovati per $I_{B_1}$, abbiamo ottenuto $V_{a_1} = (24 \pm 5)$ \si{V} e $R_{L_1} = (3.0 \pm 0.5)$ \si{k\ohm} da cui abbiamo poi ricavato la conduttanza $g_1 = (0.33 \pm 0.06)$ \si{\milli\siemens}. Per la corrente di base $I_{B_2}$ invece abbiamo trovato $V_{a_2} = (23 \pm 4)$ \si{V} e $R_{L_2} = (1.6\pm 0.3)$ \si{k\ohm} da cui abbiamo poi ricavato $g_2 = (0.63 \pm 0.12)$ \si{\milli\siemens}. È stato infine possibile calcolare il guadagno di corrente $\beta = (1.3 \pm 1.1) \cdot 10^{2}$, a prova del fatto che il BJT in questa configurazione si comporta come un amplificatore di corrente.