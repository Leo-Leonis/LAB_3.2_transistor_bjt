\section{Dati raccolti} \label{ch:data}
\noindent I valori misurati delle correnti di base misurati sono $I_{B_1} = -(0.05 \pm 0.02)$ \si{\milli\ampere} e $I_{B_2} = -(0.10 \pm 0.02)$ \si{\milli\ampere}. I dati misurati relativi alle caratteristiche $I_C$-$V_{CE}$ sono riportati in \autoref{tab:data}. 
\begin{table}[htb]
    \centering
    \begin{tabular}{||c|c||c|c||c|c||}
        \hline \hline
        \multicolumn{6}{||c||}{\textsc{caratteristiche i-v in uscita}} \\
        \hline \hline
         & & \multicolumn{2}{c||}{$I_{B_1} = -50$ \si{\micro\ampere}} & \multicolumn{2}{c||}{$I_{B_2} = -100$ \si{\micro\ampere}}\\
         \hline
        $|V_{CE}|$ (\si{V}) & f.s. & $|I_{C_1}|$ (\si{\milli\ampere}) & f.s. & $|I_{C_2}|$ (\si{\milli\ampere}) & f.s.\\ \hline
        \csvreader[
            head to column names,
            separator=tab,
            % table head = $V_\text{MM}$ & f.s. & $V_\text{osc}$ & f.s.\\\hline
             late after line=\\
            % table foot = \hline \hline
            % table head = \toprule
        ]{data/data.csv}{}
        {\Volt  $\text{ }\pm$ \VoltErr & \VoltFs & \FCurr  $\text{ }\pm$ \FCurrErr & \CurrFs & \HCurr  $\text{ }\pm$ \HCurrErr & \CurrFs}\hline\hline
    \end{tabular}
    \caption{Dati relativi alle caratteristiche in uscita del BJT per due differenti correnti di base $I_{B_1} = -50$  \si{\micro\ampere} e $I_{B_2} = -100$ \si{\micro\ampere}.}
    \label{tab:data}
\end{table}

\section{Valutazione delle incertezze} \label{ch:err}
\subsection{Oscilloscopio}
Per calcolare l’incertezza $\sigma_V$ associata ad ogni singola misura di tensione $V$ effettuata con l’oscilloscopio, abbiamo utilizzato la formula: 
\begin{equation*}
    \sigma_V = \sqrt{{\sigma_L}^2 + {\sigma_Z}^2 + {\sigma_C}^2 }
\end{equation*}
con $\sigma_C = V \cdot \alpha$, dove $\alpha$ è la precisione dichiarata dal costruttore (che in questo caso è 3\%), mentre $\sigma_L$ e $\sigma_Z$ rappresentano rispettivamente l’errore sulla lettura e l’errore sullo zero, entrambi da determinare secondo la seguente relazione:
\begin{equation*}
    \sigma_L = \sigma_Z = \dfrac{V_\text{f.s.}}{5} \cdot N_\text{t.a.}
\end{equation*}
in cui $V_\text{f.s.}$ rappresenta il fondoscala utilizzato per la singola misura e $N_\text{t.a.}$ indica il numero di tacchette apprezzabili.

\subsection{Multimetro}
Per quanto riguarda l'incertezza $\sigma$ da associare ad ogni misura effettuata con il multimetro digitale, sono state sfruttate le specifiche riportate in \autoref{tab:multimetro}; dunque ad ogni misura $x$ bisogna associare un'incertezza $\sigma_x$ secondo l'espressione
\begin{equation*}
    \sigma_x = x \cdot \text{prec.} + \text{risoluzione} \cdot \text{digits} 
\end{equation*}
con $\text{prec.}$, $\text{risoluzione}$ e $\text{digits}$ da estrapolare dalle colonne in \autoref{tab:multimetro}.

% bisogna includere di section alternativo perché hyperref non riesce a leggere le cose in math mode
\subsection[Incertezze di g e di beta]{Incertezze di $g$ e di $\beta$}
Per stimare l'incertezza associata alle misure di $g$ e $\beta$ sono state usate le seguenti formule:
\begin{equation*}
    \Delta g = \dfrac{\Delta R_L}{{R_L}^2}
\end{equation*}
\begin{equation*}
    \Delta\beta = \left(\dfrac{\Delta I_{C_2} + \Delta I_{C_1}}{ I_{C_2} - I_{C_1}} + \dfrac{\Delta I_{B_2} + \Delta I_{B_1}}{ I_{B_2} - I_{B_1}}\right)
\end{equation*} 
