\begin{circuitikz}[scale = 1, every node/.style={scale=0.9}]
% \foreach \i in {0,...,6} {
%     \draw [very thin,white!80!black] (\i,0) -- (\i,6)  node [below] at (\i,0) {$\i$};
% }
% \foreach \i in {0,...,6} {
%     \draw [very thin,white!80!black] (0,\i) -- (6,\i) node [left] at (0,\i) {$\i$};
% }
\draw % cornice (in basso a sx in senso orario)
    (0,1)
    to[battery2, l_=$-5\si{\volt}$, invert] (0,6)
    -- (6,6)
    -- (6,3)
    to [pR, name=dx, mirror, wiper pos=0.2] (6,1)
    -- (0,1)
;
\draw % potenziometro di sx (dal basso)
    (1.5,1) node[circ]{}
    -- (1.5,3.75)
    to[pR, l=1\si{\kilo\ohm}, name=sx, mirror, wiper pos=0.8] (1.5,6)
    node[circ]{}
;
\draw % bjt (dal basso)
    (4.45,1) node[circ]{}
    -- (4.45,2.15)
    node[pnp, anchor=E, xscale=-1, yscale=-1, tr circle] (bjt) {}
    (bjt.base) -| (dx.wiper) % da B a dx
;
\draw % da C con amperometro a sx
    (bjt.collector)
    to[smeter, t=A, l=$I_C$] ++(0,1.55)
    |- (sx.wiper)
;
\draw
    (3,1) coordinate(BASSO)
    to[oscope, l=$V_{CE}$, *-*] (sx.wiper -| BASSO)  
;

\draw (dx) node[left=3pt] {100\si{k\ohm}};
\node [circ] at (bjt.circle C){};
\draw (bjt.circle C) node[above left] {$C$};
\node [circ] at (bjt.circle E){};
\draw (bjt.circle E) node[below left] {$E$};
\node [circ] at (bjt.circle base){};
\draw (bjt.circle base) node[above right] {$B$};
\end{circuitikz}