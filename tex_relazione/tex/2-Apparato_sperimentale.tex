Lo scopo della prova è quello di misurare le caratteristiche I-V in uscita di un transistor BJT PNP in configurazione a emettitore comune, per due differenti valori della corrente di base.

\subsection{Materiali e strumenti}
Riportiamo adesso l'elenco dei materiali e degli strumenti utilizzati durante l’esperienza (tra parentesi il modello utilizzato):
\begin{multicols}{2}
    \begin{itemize}
         \item Generatore di tensione (\verb|ISO-TECH IPS 3303|)
        \item Potenziometro da 1\si{\kilo\ohm}
        \item Potenziometro da 100\si{\kilo\ohm}
        \item Transistor PNP al Si (\verb|BC 212|)
        \item Breadboard (\verb|K&H AD-13|)
        \item Oscilloscopio analogico (\verb|ISO-TECH ISR 622|)
        \item Multimetro digitale (\verb|Fluke 77 IV|)
        \item [\vspace{\fill}]
    \end{itemize}
\end{multicols}

\begin{table}[h]
    \centering
    \begin{tabular}{||c|c|c|c||}
        \hline\hline
        \multicolumn{4}{||c||}{\textsc{multimetro} \ \texttt{Fluke 77 IV}}\\
        \hline\hline
        fondo scala & risoluzione & prec. & digits \\\hline
        60.00 \si{\milli\ampere} & 0.01 \si{\milli\ampere} & \multirow{4}{*}{1.5\%} & \multirow{4}{*}{2}\\
        400.0 \si{\milli\ampere} & 0.1 \si{\milli\ampere} &  & \\
        6.000 \si{\ampere} & 0.001 \si{\ampere} &  & \\
        10.00 \si{\ampere} & 0.01 \si{\ampere} &  & \\
        \hline\hline
    \end{tabular}
    \caption{Specifiche tecniche del multimetro \ \emph{\texttt{Fluke 77 IV}} relative alle misurazioni di corrente in corrente continua. La colonna \emph{prec.} indica la precisione e \emph{digits} indica il numero da aggiungere all'ultima cifra significativa della misura.}
    \label{tab:multimetro}
\end{table}

\subsection{Svolgimento dell'esperienza}
\begin{figure}[htb]
    \centering
    \begin{subfigure}{.4\textwidth}
    \centering
        \begin{circuitikz}[scale = 1, every node/.style={scale=0.9}]
% \foreach \i in {0,...,6} {
%     \draw [very thin,white!80!black] (\i,0) -- (\i,6)  node [below] at (\i,0) {$\i$};
% }
% \foreach \i in {0,...,6} {
%     \draw [very thin,white!80!black] (0,\i) -- (6,\i) node [left] at (0,\i) {$\i$};
% }
\draw % cornice (in basso a sx in senso orario)
    (0,1)
    to[battery2, l_=$-5\si{\volt}$, invert] (0,6)
    -- (6,6)
    -- (6,3)
    to [pR, name=dx, mirror, wiper pos=0.2] (6,1)
    -- (0,1)
;
\draw % potenziometro di sx (dal basso)
    (1.5,1) node[circ]{}
    -- (1.5,3.5)
    to[pR, l=1\si{\kilo\ohm}, name=sx, mirror, wiper pos=0.7] (1.5,6)
    node[circ]{}
;
\draw % bjt (dal basso)
    (3,1) node[circ]{}
    -- (3,3)
    node[pnp, anchor=E, xscale=-1, yscale=-1, tr circle] (bjt) {}
    (bjt.collector) |- (sx.wiper) % da C a sx
;
\draw % amperometro (dal bjt in alto)
    (bjt.base)
    -| (4.75,3.6)
    to[smeter, t=A, l=$I_B$] (4.75,2.5)
    |- (dx.wiper)
;
\draw (dx) node[left=3pt] {100\si{k\ohm}};
\node [circ] at (bjt.circle C){};
\draw (bjt.circle C) node[above left] {$C$};
\node [circ] at (bjt.circle E){};
\draw (bjt.circle E) node[below left] {$E$};
\node [circ] at (bjt.circle base){};
\draw (bjt.circle base) node[above right] {$B$};
\end{circuitikz}
        \caption{Schema del circuito realizzato per il regolamento della corrente di base al punto \emph B.}
        \label{fig:circ_1}
    \end{subfigure}
     \hspace{1cm}
    \begin{subfigure}{.4\textwidth}
        \centering
        \begin{circuitikz}[scale = 1, every node/.style={scale=0.9}]
% \foreach \i in {0,...,6} {
%     \draw [very thin,white!80!black] (\i,0) -- (\i,6)  node [below] at (\i,0) {$\i$};
% }
% \foreach \i in {0,...,6} {
%     \draw [very thin,white!80!black] (0,\i) -- (6,\i) node [left] at (0,\i) {$\i$};
% }
\draw % cornice (in basso a sx in senso orario)
    (0,1)
    to[battery2, l_=$-5\si{\volt}$, invert] (0,6)
    -- (6,6)
    -- (6,3)
    to [pR, name=dx, mirror, wiper pos=0.2] (6,1)
    -- (0,1)
;
\draw % potenziometro di sx (dal basso)
    (1.5,1) node[circ]{}
    -- (1.5,3.75)
    to[pR, l=1\si{\kilo\ohm}, name=sx, mirror, wiper pos=0.8] (1.5,6)
    node[circ]{}
;
\draw % bjt (dal basso)
    (4.45,1) node[circ]{}
    -- (4.45,2.15)
    node[pnp, anchor=E, xscale=-1, yscale=-1, tr circle] (bjt) {}
    (bjt.base) -| (dx.wiper) % da B a dx
;
\draw % da C con amperometro a sx
    (bjt.collector)
    to[smeter, t=A, l=$I_C$] ++(0,1.55)
    |- (sx.wiper)
;
\draw
    (3,1) coordinate(BASSO)
    to[oscope, l=$V_{CE}$, *-*] (sx.wiper -| BASSO)  
;

\draw (dx) node[left=3pt] {100\si{k\ohm}};
\node [circ] at (bjt.circle C){};
\draw (bjt.circle C) node[above left] {$C$};
\node [circ] at (bjt.circle E){};
\draw (bjt.circle E) node[below left] {$E$};
\node [circ] at (bjt.circle base){};
\draw (bjt.circle base) node[above right] {$B$};
\end{circuitikz}
        \caption{Schema del circuito realizzato per la misura della caratteristica in uscita del transistor.}
        \label{fig:circ_2}
    \end{subfigure}
    \caption{I due circuiti realizzati nello svolgimento dell'esperienza. $V_{CE}$ indica l'oscilloscopio, mentre $I_B$ e $I_C$ indicano il multimetro.}
    \label{fig:circuiti}
\end{figure}

L'esperimento è composto da due fasi: nella prima fase, dopo aver realizzato il circuito rappresentato in \autoref{fig:circ_1}, è stato necessario fissare la corrente di base a $-50$ \si{\micro\ampere}, agendo sul potenziometro da 100 \si{k\ohm}; nella seconda fase, utilizzando il circuito rappresentato nella \autoref{fig:circ_2}, sono stati misurati contestualmente i valori della corrente $I_C$ con il multimetro e della tensione $V_{CE}$ con l'oscilloscopio facendo variare la corrente in uscita nel collettore tramite il potenziometro da 1 \si{k\ohm}.

Infine, è stato ripetuto l'esperimento per un valore della corrente di base di $-100$ \si{\micro\ampere}.